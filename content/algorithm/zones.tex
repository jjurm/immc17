
The next alternative we considered was to choose several of our factors, that have the biggest influence on productivity and divide the world into zones, so that a single zone is homogeneous according to these primary factors and that factors of each zone would cover the whole world. We could then search through the zones, in order to find the one with the best conditions according to the primary factors. 

This method gives us an approximation of which zone should be the best according to the primary factors. Then we could "zoom in" to the found zone and examine the cities located in it based on secondary factors. 

However, we have found some flaws in this approach. First of all, what should be considered as a primary factor? Even if we created two sets of factors labeled more important and less important their significance could vary depending on attendant input. Moreover, assuming the relevance of each factor would not be influenced by attendant input then still there would be more important factors which could not cover the entire world in zones, for instance the daylight factor, or altitude factor. 

Furthermore, if we decided to pick the factors with the ability to cover the whole world as the more important ones, then there is still a chance that after applying the second round of factors there would be a place in the world that would be more ideal with respect to both rounds of sub-models. For these reasons we chose to not use this model.