
So far, the only criteria we have considered are related to productivity. This method of finding the perfect spot for holding the meeting is great from a mathematical perspective, but in real life, there are other criteria that may influence the decision besides maximum productivity. The most significant non-productivity based criterion is the overall cost.

Determining the overall cost of such an event is rather problematic, because there are many variables that can differ from place to place, such as accomodation, food, travel etc.

One way to globally compare cities and countries in this aspect is by using the Cost of Living Plus Rent Index, which is an estimation of consumer goods prices including rent in the city comparing to New York City. This means that for New York City, the index should be 100(\%). If another city has, for example, Cost of Living Plus Rent Index of 120, it means in average that city is 20\% more expensive than New York City. This means, that to calculate the overall cost of living in a city ($O$), we would need to multiply the Cost of Living Plus Rent Index ($C$) divided by 100 (because it is given in $\%$ relative to NYC) by the overall estimated cost of living in New York City (1,054.81€): $O=\frac{C}{100}\cdot 1054,81$.
 
Now our aim is to rank the same cities that we used in the productivity-based ranking based on the Cost of Living Plus Rent Index. To achieve this we are going to use the Numbeo database, but due to lack of data for some cities we are going to consider the Cost of Living Plus Rent Index of the country. To determine the index of those cities, which are not in the Numbeo database nor their country is in the database, we are going to use the same method as explained in section 5.8 - Quality of life index. After allocating an index to every city we calculate the overall cost of living ($O$) using the formula above. 

We need to realise, that cost is just a secondary criterion, which to some of us can matter more than to others. We could say, that everyone values their productivity differently. To find the ideal place to hold the meeting, the organisers need to determine the "cost" of 1\% of productivity. By doing that, we can calculate the "cost" of every productivity for each city, which means that we can compare it with the overal cost ($O$). For that we are going to use a parameter ($p$), which has to be set by the organisers. The productivity cost($C_p$) is calculated  the following way:
$$C_p=P\cdot p$$

The place to hold the meeting will be the one, which has the highest productivity for the lowest price, so in other words, the city which has the largest difference between productivity cost and overall cost ($C_p-O$).

