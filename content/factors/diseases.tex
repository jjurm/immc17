
When travelling to a foreign country, we need to consider the differences beetween our destination and our home country from the health perspective as well. These differences are present in many different areas. For instance drastic change in cuisine can cause severe digestion problems or certain areas may be more violent etc. 

Being in a perfect shape is very important in order to sustain a high level of productivity. That is why our model should also take this factor into account. The ideal way to do this is to identify diseases which have an incubation period of 3 days or less (so that the disease influences the productivity) and calculate the probability of someone getting infected.

The most reliable source for International comparisons is the World Health Organization because it is standardized for cross cultural comparisons. It is also reviewed annually, which often makes it more current than individual country data that can take years to compile.

The problem with incorporating diseases into our model is the calculation of the probability of someone getting infected. Even though we have some data available (appendix excel document), for many diseases it is incomplete, especially in third world countries and that could cause serious issues when determining the best place to hold the meeting. In addition to this, we are well aware of the fact that only a small number of diseases have an incubation period of 3 days or less, so this factor would not have a huge impact on the productivity either way. That is why we have decided to exclude it from our model.