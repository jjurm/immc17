
Bad rainy weather in general induces a negative mood, so naturally, we feel that our work productivity should be decreased by its effect. This fact is supported by a recent study, which shows, that about 82\% of the respondents thought that weather conditions would increase productivity, and about 83\% responded that bad weather conditions would decrease productivity.

However, the same study suggests, that on a sunny day people are tempted to think about doing outdoor activities and they tend to be less focused on their work. This hypothesis was tested by assessing worker productivity using archival data from a Japanese bank’s home-loan mortgage-processing line. The data includes information on the line from the rollout date, June 1, 2007 until December 30, 2009, a 2.5-year time period. In total 598,393 transactions made by 111 workers were examined. 

According to the results of this study, a one-inch increase in rain is related to a 1.363\% decrease in time required for each transaction, which leads to better productivity. Also, the error rates were very small (less than 3\%) and the variation was  little, so we can consider the decreased time the only change in productivity. 

To determine the increase in rain, comparison between the precipitation in the home country ($p_H$) and the destination ($p_D$) is required. We used historical precipitation data per month derived from the Climate Research Unit (Mitchell et al, 2003), aggregated by countries. The increase in rain is calculated as the difference between ($p_H$) and ($p_D$) divided by 2.54 (conversion from cm to inch): $\frac{(p_D - p_H)}{2.54}$.

Since we know, that a one-inch increase in rain results in a 1.363\% decrease in time, a one-inch increase leads to a 1.363\% increase in productivity. The study suggests that the functionality between the increased amount of rain and productivity is linear, which means, that productivity can be calculated the following way:

$$P=\frac{(p_D - p_H)}{2.54}\cdot 1.01363$$


Precipitation is a factor, that reflects the entire climate of a certain area. It is highly unlikely, that someone coming from a different climate would get used to it over the course of three days, so this aspect is not considered in our model.

