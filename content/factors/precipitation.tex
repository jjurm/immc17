
Bad rainy weather in general induces a negative mood, so naturally, we feel that our work productivity should be decreased by its effect. This fact is supported by a recent study, which shows, that about 82\% of the respondents thought that weather conditions would increase productivity, and about 83\% responded that bad weather conditions would decrease productivity.

However, the same study suggests, that on a sunny day people are tempted to think about doing outdoor activities and they tend to be less focused on their work. This hypothesis was tested by assessing worker productivity using archival data
from a Japanese bank’s home-loan mortgage-processing line. The data includes information on the line from the rollout date, June 1, 2007 until December 30, 2009, a 2.5-year time period. In total 598,393 transactions made by 111 workers were examined. 

According to the results of this study, a one-inch increase in rain is related to
a 1.363\% decrease in time required for each transaction, which leads to better productivity. Since productivity is not measured only by quantity, but by quality as well, a standard error had to be calculated resulting in a 0,6068\% daily error. 

Precipitation is a factor, that reflects the entire climate of a certain area. It is higly unlikely, that someone coming from a different climate would get used to it over the course of three days, so this aspect is not considered in our model.

