Another aspect of a local climate that can heavily influence the ability of an individual to work in a particular environment is the relative humidity of the surrounding air. Humidity directly influences the heat index, a metric that determines the human-perceived temperature, given objective air temperature and relative air humidity. In order to implement humidity into our model, we only need to calculate the heat index for a given location in a particular time and use it as an argument for our temperature model instead of objective temperature. Heat index $HI$ can be approximated using several formulas. We decided to use the most widely used one. This formula uses the Fahrenheit scale, so we also need to convert the dry-bulb temperature into Fahrenheit and the resulting value back to Celsius.

$$HI_{\degree F}= c_1+c_{2}t_{\degree F}+c_{3}R+c_{4}t_{\degree F}R+c_{5}t_{\degree F}^2+c_{6}R^2+c_{7}t_{\degree F}^{2}R+c_{8}t_{\degree F}R^2+c_{9}t_{\degree F}^{2}R^2$$

%$$HI_{\degree C}=\left(HI_{\degree F}\left(\dfrac{9}{5}\cdot t+32\right)-32 \right) \cdot \dfrac{5}{9}$$

%$$HI_{\degree C}=\left( c_1+c_{2}\left(\dfrac{9}{5}\cdot t+32\right)+c_{3}R+c_{4}\left(\dfrac{9}{5}\cdot t+32\right) \cdot R+c_{5}\left(\dfrac{9}{5}\cdot t+32\right)^2+c_{6}R^2+c_{7}\left(\dfrac{9}{5}\cdot t+32\right)^{2}R+ \right.$$
%\begin{flushright}
%$\left. +c_{8}\left(\dfrac{9}{5}\cdot t+32\right) \cdot R^2+c_{9}\left(\dfrac{9}{5}\cdot t+32\right)^{2}R^2 -32 \right) \cdot \dfrac{5}{9}$
%\end{flushright}

\begin{align*}
HI_{\degree F} &- \text{heat index in degrees Fahrenheit}\\
t_{\degree F} &- \text{dry-bulb temperature in degrees Fahrenheit}\\
R &- relative humidity expressed in percents
\end{align*}
\begin{align*}
c_1 &= -42.379 & c_4 &= -0.22475541 & c_7 &= 1.22874 \cdot 10^{-3}\\
c_2 &= 2.04901523 & c_5 &= -6.83783 \cdot 10^{-3} & c_8 &= 8.5280 \cdot 10^{-4}\\
c_3 &= 10.14333127 & c_6 &= -5.481717 \cdot 10^{-2} & c_9 &= -1.99 \cdot 10^{-6}
\end{align*}

Now we can use the resulting value of $HI_{\degree C}$, heat index in degrees Celsius, as an argument for the our temperature model.

Unfortunately we haven't managed to find sufficient data with average humidity values. Hence the temperature model only uses the real temperature values instead of calculated heat indexes.
