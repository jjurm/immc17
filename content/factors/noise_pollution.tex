There are a few distinct ways to interpret what noise is. Noise can mean unwanted fluctuations in electrical signal due to seemingly random movement of electrons. This definition is used in audio processing where different algorithms are applied to smoothen this undesired event. Noise might also be defined as irrelevant or meaningless data occurring along with desired information. In any case, noise means something rather unpleasant which we have to somehow deal with. Usually, there are software solutions that filter these undesirable inputs/outputs, but for certain noises we can not rely on software, we have to entrust our brain.


An Oxford University study \cite{Stansfeld2003} points out how noise pollution plays a negative role in various aspects of our lives. There is a number of psychological, mental and cardiovascular symptoms which occur mainly due to longer disruptive sound exposure. For our model however, our main concern should be the cognitive effect which is directly linked with annoyance and disruption. The more disrupted someone is, the less they are able execute a task and ergo they become less productive. 


Another type of "noise" that disturbs us moves in light waves. Light pollution is believed to be the main cause for considerable drop in Melatonin, which also means worse and shifted sleep patterns and even is connected with cancer, namely breast cancer \cite{Chepesiuk2009}. Particularly the effect on sleep might noticeably shape one's productivity.


According to the Weber-Fechner law there is a logarithmic relation between biological stimulus and the perceived sensation. We can notice this phenomenon when we push a finger against our palm. We are much more sensitive to the change of force in reaction to slight taps than to forceful pushes. This makes perfect sense since our sensitivity is the derivative of this logarithmic function. We could consider some relation between annoyance and brain sensitivity which would lead to the dependency between sound wave energy and annoyance, nevertheless there is a lack of sufficient data when it comes to average amount of decibels per city or per country. 


Moreover, figuring out the model for light pollution and productivity would lead to similar problems. Even though there is a great amount of research done on sleep deprivation and Melatonin levels, there is virtually no data telling us about night light in different areas of the world. 


There is a company called Numbeo which crowdsourseces data about countries/cities. One of the factors they include in their API is noise and light pollution. The API provides a number ranging between -2 and 2 which tells us to what extent people are annoyed by the observed factor in their respective countries/cities. Using this input we can model the influence on one's productivity. Since the scale we get from the API is linear and the productivity scale is exponential, the function which maps the API output to productivity is the following function of $x$, defined on $ x\in \langle -2,2\rangle$:

$$P = f(x) = p^{\frac{2-x}{4}}$$

\noindent where $p$ is a parameter which tells the productivity in the worst case scenario  (when $x=-2$) thus $ p\in \langle -0,1\rangle$. Furthermore, our function always fulfills $f(2) = 1$ since for any place where the noise and light pollution causes no harm we should consider a maximal productivity of $P = 1$.


There are no evident sources that prove how the effect of these factors on individuals changes over time. Of course, in the long run they can mean severe health risks, however in a scope of 3 days this influence is practically undetectable.





