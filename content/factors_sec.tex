Moving someone to an international meeting we have to consider a number of factors that might have a subsequent impact on their productivity. Several factors are fairly apparent due to the severity and scope of their influence. Some of these more evident elements are jet lag, temperature difference, or altitude level. Nonetheless, further examination of more subtle aspects is crucial in order to maximize the credibility of our model. Factors such as noise pollution, crime, or rainfall all belong to this group of delicate aspects. 

In this section, we will firstly examine which factors to consider in our model. Secondly, we will design a model for each of them. And lastly, we are going to estimate how the effect of an observed sub-model changes over the course of 3 days.

Our sub-models heavily rely on data and research. These sources are sufficient enough to evaluate most relations and properties of the desired models, however, in some cases, there is a lack of adequate information for us to precisely design the effect on one's productivity. For instance, one resource \cite{daylight2002} suggests that working under natural light creates an increase in productivity of about 21\%. This means that we should consider a function of decreasing productivity for the amount of time spent before sunrise, or after sunset. Yet we can only roughly estimate the scale to which this decrease in lighting will actually take a toll on productivity. That is the reason for our inclusion of parameters in most of the models.

The primary function of these parameters is to create models which are easily customizable. Customization means than any consequent research which goes deeper into the dependency of our factors and productivity can be easily included in our sub-models. Furthermore, using these parameters we can involve attendant input based on their preferences. Information such as their age, country of origin, and rates to which they believe certain factors affect them can be easily obtained by simple psychological surveys. This way we will be able to tailor each model to predilections of our attendants. The parameters we use in the algorithm are based on our research of each influencing aspect.

Each sub-model is a parametrized function returning the adjusted productivity based on a particular criterion.

\begin{table}[h!]
\centering
\begin{tabular}{c|c}
Parameters & Arguments \\
\hline \hline
attendant & destination \\
meeting date & day index \\
%\hline
\end{tabular}
\caption{parameters and arguments of a model function}
\label{table:model-function}
\end{table}

Here, by the attendant parameter, we mean every information that we decide to investigate (attendant's origin, age, etc.).

Our methodology when it comes to estimating these parameters is based on our own experiences, research, and testing the model. 

\subsection{Jet Lag}
\import{content/factors/}{jetlag}

\subsection{Distance}
\import{content/factors/}{distance}

\subsection{Altitude}
\import{content/factors/}{altitude}

\subsection{Daylight}
\import{content/factors/}{daylight}

\subsection{Temperature}
\import{content/factors/}{temperature}

\subsection{Air humidity}
\import{content/factors/}{humidity}

\subsection{Noise and light pollution}
\import{content/factors/}{noise_pollution}

\subsection{Quality of life}
\import{content/factors/}{life_quality}

\subsection{Diseases}
\import{content/factors/}{diseases}

\subsection{Precipitation}
\import{content/factors/}{precipitation}
